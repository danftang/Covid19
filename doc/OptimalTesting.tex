\documentclass{article}

\usepackage{arxiv}

\usepackage[utf8]{inputenc} % allow utf-8 input
\usepackage[T1]{fontenc}    % use 8-bit T1 fonts
\usepackage{hyperref}       % hyperlinks
\usepackage{url}            % simple URL typesetting
\usepackage{booktabs}       % professional-quality tables
\usepackage{amsfonts}       % blackboard math symbols
\usepackage{nicefrac}       % compact symbols for 1/2, etc.
\usepackage{microtype}      % microtypography
\usepackage{lipsum}		% Can be removed after putting your text content
\usepackage{amssymb,amsmath}
\usepackage{listings}
\usepackage{graphicx}
\usepackage{subfig}
%\usepackage{apacite}

\title{Optimal testing schedule for mobile-phone contact-tracing}

%\date{September 9, 1985}	% Here you can change the date presented in the paper title
%\date{} 					% Or removing it

\author{
  Daniel Tang\\
  Leeds Institute for Data Analytics\thanks{This project has received funding from the European Research Council (ERC) under the European Union’s Horizon 2020 research and innovation programme (grant agreement No. 757455)}\\
  University of Leeds\\
  Leeds, UK\\
  \texttt{D.Tang@leeds.ac.uk} \\
  %% examples of more authors
  %% \AND
  %% Coauthor \\
  %% Affiliation \\
  %% Address \\
}

\begin{document}
\maketitle

\begin{abstract}
With the recent announcement\cite{applegoogle} that Apple and Google will introduce a contact-tracing API to iOS and Android, and later add contact tracing functionality directly to their OS's, it seems increasingly likely that contact tracing via a smart phone will form an important part of the effort to manage the COVID-19 pandemic and prevent resurgences of the disease after an initial outbreak.

In this paper we develop an optimal testing schedule 
\end{abstract}

% keywords can be removed
\keywords{COVID-19, SARS-CoV-2}

Suppose we have a number of tests $1...n$ that give us information about whether a person is infected with a disease. The characteristics of the $n^{th}$ test are defined by it's specificity, $P_n(-|\bar{i})$, which gives the probability that a non-infected person will test negative if tested with test $n$ , and it's sensitivity $P_n(+|i,\Delta t)$ which gives the probability that a person infected due to an exposure at a time $\Delta t$ before the test will test positive if tested with test $n$.

Suppose the tests are used in the context of contact tracing and we assess the probability that person A is infected as $P_A(i)$. Suppose person B has been identified as having had a close contact with person A at a time $\Delta t$ ago.

We don't know whether A infected B or B infected A. We know B's other close contacts.

Each test has a financial cost $C_{Tn}$. We also attach a cost per day to isolating a person $C_I$ and a cost per day of allowing an infected person to be not isolated $C_{\bar{I}|i}$.

Given this information, what is the schedule of testing/isolation of person B that minimises the expected cost.

We first need to calculate the probability distribution over B's exposure time. We assume that there is a constant rate of environmental infection (via surfaces etc.) $\rho_e$ per day, a prior probability of infection $P(i)$ and a probability of transmission given a close contact with an infected person $P(t|i)$. So, given that person B has had close contacts with people other than A at times $t_1...t_m$ ago and a close contact with A at time $t_A$ we have
\[
P_{exp}(t) = \rho_e + \sum_{k=1}^m \delta(t-t_k)P(t|i)P(i) + \delta(t-t_A)P(t|i)P_A(i)
\]
So, the probability of correctly identifying a positive case with test $n$ at time $t$ is given by
\[
P_{nt}(+|i) = \int P_n(+|i,t-t_e)P_{exp}(t_e) dt_e
\]

%\bibliographystyle{unsrtnat}
\bibliographystyle{unsrturl}
%\bibliographystyle{alpha}
%\bibliographystyle{plainurl}
%\bibliographystyle{apalike} 
%\bibliographystyle{apacite}
\bibliography{references}

\end{document}
